\chapter{Research and Requirements}

To gain a more complete understanding of the current situation in Moldova, the existing problems and possible needs of the people involved, it is important to start with an analysis to identify the possible key stakeholders for this project. As previously mentioned in the literature review, a diverse group of stakeholders is essential to ensure that the currrent situation is reviewed from multiple perspectives. 

Afterwards, the next step is to utilise the chosen stakeholders to gather as much information as possible from various perspective to ensure that the project is aligned with the needs of both patients and healthcare professionals in Moldova and solves the existing problems.

\section{Stakeholder Analysis}

The student has identified the following stakeholders for the project:

\begin{itemize}
    \item A doctor working at the Republican Hospital - the main stakeholder, who will be providing guidance and feedback throughout the project on the perspective of the end users, the doctors and nurses who will be using the EHR system.
    \item The Vice-Director of CNAM (National Health Insurance Company) - the person who will be providing feedback from the perspective of the health insurance company.
    \item Senior IT staff member at CNAM - a person who also has previously worked within the Ministry of Health, and who will be providing feedback on the technical aspects of the integration between the EHR system and the health insurance system.
    \item Department head at the Republican Hospital - a person who will be providing feedback on the administrative aspects of the EHR system.
    \item Patients - both of public and private healthcare institutions, who will be providing feedback on the usability and accessibility of an EHR system prototype.
\end{itemize}

These stakeholders have been identified so that they can provide a bigger picture on the needs and requirements of the project, and to ensure that the project is aligned with the needs of the healthcare industry in Moldova from both the provider and patient perspective. 

The stakeholders have also been placed the stakeholder influence-interest grid, to help the student understand the level of influence and interest that each stakeholder has in the project:

\begin{itemize}
    \item The doctor working at the Republican Hospital - low influence, high interest
    \item The Vice-Director of CNAM - high influence, high interest
    \item Senior IT staff member at CNAM - high influence, low interest
    \item Department head at the Republican Hospital - high influence, low interest
    \item Patient - low influence, high interest
\end{itemize}

\subsection{Current situation analysis}

Following the stakeholder analysis, the student has conducted several exploratory interviews with the chosen people to provide insights into the current issues with the IT systems used in the healthcare sector. After the conclusion of the interviews, several main themes for problems and potential solutions have been identified:

\begin{enumerate}
    \item Current system is outdated, not user friendly and only accessible via Internet Explorer or legacy version of Microsoft Edge. A potential solution is to develop a new, modernized version of the existing system (thus retaining the core functionality) that is accessible via modern browsers and devices and can be augmented with additional features if necessary.
    \item Lack of interoperability between medical institutions due to a lack of a nationally-wide integrated system. A potential solution is to create a new systems where patients can upload their own medical records (such as lab tests, previous medical history, etc) and share them with any medical practitioner, regardless of the institution they work at. 
    \item Some systems are not digitalized at all, and still rely on paper-based records or very rudimentary data structures, such as the transplant registry. A potential solution is a the creation of a digitalized system, as is in the case of the transplant registry, that can be accessed by any medical practitioner in Moldova.
\end{enumerate}

After analysing the current issues and potential solutions, the student has determined that the solutions for issues \#1 and \#3 are too complex, as they require an overhaul of the system and integration with other existing systems. As such, the student has decided to focus on issue \#2, as it is the most feasible and can be implemented within the timeframe of the project. 

After the decision has been made, some changes to the stakeholders have been made. The Department head at the Republican Hospital and Senior IT staff member at CNAM have been removed from the list of stakeholders, as their feedback is no longer relevant to the project. Instead, the focus was shifted onto having more patients as stakeholders, as they will now be the main users of the system. 

The new stakeholders are as follows:
\begin{itemize}
    \item Doctor working at the Republican Hospital
    \item Doctor who previously worked at the Republican Hospital
    \item The Vice-Director of CNAM
    \item 2 Patients that are using both public and private healthcare institutions
    \item Doctor working a private healthcare institution
\end{itemize}

After the new stakeholders were identified, additional interviews were conducted to focus on the requirements for the chosen solution. The requirements have been gathered and documented in the next section.

\section{Requirements}

The student has gathered the following requirements for the project:

\subsection{Non-functional requirements}

\begin{enumerate}
    \item The system must be accessible on all modern desktop and mobile based browsers.
    \item The system must be accessible from any location by using an internet connection.
    \item When shared with the doctor via a link, the system must load within 3 seconds when accessed via a desktop browser.
    \item The system must provide a secure login mechanism for patients via MFA or e-signature.
    \item The system should allow patients to upload their own medical records in a variety of formats (PDF, DOC, etc).
    \item When shared with the doctor via a link, the system should secure the data with a unique token or PIN that expires after the specified time frame.
    \item The system could be accessible on all modern mobile devices via a mobile application.
\end{enumerate}

\subsection{Patient history requirements - doctor perspective}
\begin{enumerate}
    \item The system must provide an overview of the patient history through 3 main sections: personal information, lab tests and doctor consultations.
    \item The system must allow the doctor to only view the patient's history, not edit it.
    \item The system must allow the doctor to view blood tests in a graphical format.
    \item The system must allow the doctor to view blood tests in a numerical, tabular format.
    \item The system must allow the doctor to view the patient's history in a chronological order.
    \item The system must display the doctor consultation and every lab test,except for blood tests, in a free text or document format.
    \item For blood test results, the system should display the normal range values for each test.
\end{enumerate}

\subsection{Patient personal data requirements}
\begin{enumerate}
    \item The system must allow the patient to include their personal information, such as name, date of birth, address, allergies, etc.
    \item The system must allow the patient to include their insurance information.
    \item The system must allow the patient to enter vitals information, such as height, weight, blood pressure, etc.
    \item When multiple vital entries are made, the system should display a historical graph of the patient's vitals.
\end{enumerate}

\subsection{Patient medication}
\begin{enumerate}
    \item The system must allow the patient to enter their current medication.
    \item The system must allow the patient to enter their past medication.
    \item The system must allow the patient to enter the dosage and frequency of the medication.
    \item The system must allow the patient to enter the reason for taking the medication.
    \item The system must allow the patient to enter the start and end date of the medication.
    \item The system must allow the patient to enter the doctor who prescribed the medication.
    \item When adding medication, the system should have 2 options: add a simplified version of the medication or add a detailed version of the medication.
    \item When choosing the simplified version, the system should allow the patient to just add the name, dosage and duration of the medication.
    \item When choosing the detailed version, the system should allow the patient to add the name, dosage, duration, reason for taking the medication, doctor who prescribed the medication, and any additional notes.
    \item After entering the medication, the system could allow the patient to track the medication intake.
    \item The system could send reminders to the patient to take the medication by using push notifications.
\end{enumerate}

\subsection{Patient shareable link}
\begin{enumerate}
    \item The system must allow the patient to generate a shareable link to their medical records.
    \item When creating the shareable link, the system must allow the patient to set an expiration date for the link.
    \item When creating the shareable link, the system should allow the patient to set an access PIN or token for the link.
    \item When create the shareable link, the system should allow the patient to select which records to share with the doctor.
\end{enumerate}