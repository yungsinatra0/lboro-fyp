\chapter{Research and Requirements}

This chapter will outline the stakeholder analysis process, followed by requirements gathering. A diverse group of stakeholders was chosen to ensure multiple perspectives of the healthcare sector are represented and so that a comprehensive list of requirements is created. At the of this chapter, a list of the most important requirements will be presented.

\section{Stakeholder Analysis}

The initial stakeholders identified included:

\begin{itemize}
    \item Doctors and other medical staff working in hospitals
    \item Department head of a republican hospital
    \item IT staff members in hospitals
    \item CNAM staff members
    \item Ministry of Health staff
    \item Patients 
\end{itemize}

These stakeholders were mapped on an influence-interest grid (which will be discussed in section~\ref{sec:stakeholders}) to help better understand the level of influence and interest that each stakeholder has in the project and prioritise the engagement with them. The grid can be seen in figure~\ref{fig:stakeholder_analysis1}.

\begin{figure}[ht]
    \centering
    \includegraphics[scale=0.5]{Stakeholder_analysis1.png}
    \caption{Chosen stakeholders in the stakeholder influence-interest grid}\label{fig:stakeholder_analysis1}
\end{figure}

\subsection{Current situation analysis}

Following the analysis, exploratory interviews were conducted with available stakeholders, excluding CNAM and Ministry of Health staff members who were not available for interviews. These discussions revealed three main challenges in Moldova's healthcare sector:

\begin{enumerate}
    \item The current EHR system is outdated and only accessible via Internet Explorer or legacy version of Microsoft Edge. A potential solution was proposed to develop a modernised version of the existing system, retaining the core functionality. 
    \item The lack of interoperability between medical institutions due to an absence of a nationally-wide integrated system. A potential solution was proposed to create a digital patient archive, where users can upload their own medical records (such as lab tests, previous medical history, etc) and share them with any healthcare practitioner. 
    \item The lack of digitalisation of systems that still rely on paper-based records, such as the national transplant registry. A potential solution was proposed to create a digital transplant registry, that can be accessed by any healthcare practitioner in Moldova.
\end{enumerate}

After evaluating the feasibility of the proposed solutions, it was decided for the project to focus on issue \#2, as the other two solutions were determined to be too complex for the constraints of an academic project. The decision was supported by the fact that the chosen solution would be independent from existing healthcare systems, allowing for a more straightforward and less complex implementation. 

Consequently, the stakeholder list has been updated to reflect the changed focus of the project:
\begin{itemize}
    \item Doctors working in hospitals and clinics
    \item Ministry of Health staff
    \item CNAM staff
    \item Patients
    \item Other medical staff members
    \item CNPDCP staff
\end{itemize}

At the same time, the stakeholder influence-interest grid has been updated to reflect the changes in the project focus, which is illustrated in figure~\ref{fig:stakeholder_analysis2}.

\begin{figure}[ht]
    \centering
    \includegraphics[scale=0.5]{Stakeholder_analysis2.png}
    \caption{Updated stakeholders in the stakeholder influence-interest grid}\label{fig:stakeholder_analysis2}
\end{figure}

\clearpage

\section{Requirements}

Additional interviews were conducted to focus on specific requirements for the chosen solution. Enough information was gathered from the other stakeholders to identify the core requirements for the project. Staff members at CNAM, the Ministry of Health and CNDCP were still not available for interviews, so instead legislation and regulations on their websites will be reviewed \parencite{CNAM,CNPDCP,ministry}. 

All of the gathered requirements can be found in appendix~\ref{sec:requirements}, but the most important ones have been summarised in the table below:

\begin{table}[h!]
    \centering
    \begin{tabular}{|c|c|p{0.65\textwidth}|}
    \hline
    \textbf{ID} & \textbf{Category}                  & \textbf{Requirement}                                                                \\ \hline
    1   & Document upload & When a the lab category is selected during record creation, the system must allow the user to enable AI processing.       \\ \hline
    2   & Document upload & If lab extraction was enabled, the system must display the extraction results to the user before sending to the database. \\ \hline
    3   & Document upload & The system must allow patients to upload their own medical records in a variety of formats (PDF, DOC, etc). \\ \hline
    4  & Personal cabinet & The system must display the patient's history in a chronological order in a table format. \\ \hline
    5   & Personal cabinet & The patient personal cabinet must provide an overview of the patient's history through 3 main sections: personal information, lab tests, and doctor consultations. \\ \hline
    6  & Personal cabinet & The system must allow patients to add their own personal information, such as name, date of birth, or address. \\ \hline
    7  & Personal cabinet & The system must allow patients to add their own allergies, vaccinations and medication. \\ \hline
    8   & Personal cabinet & When multiple vital and lab entries are made, the system should display a historical graph of the results. \\ \hline
    9   & Shareable link & The system must allow the patient to generate a shareable link to provide access to their medical records. \\ \hline
    10   & Doctor view & When shared with the doctor, the system must allow the doctor to only view the patient's history, not edit it. \\ \hline
    \end{tabular}
    \caption{Summary of the Most Important Requirements}
\end{table}
    
    
    
    