\chapter{Research and Requirements}

To gain a more complete understanding of the current situation in Moldova, the existing problems and possible needs of the people involved, it is important to start with an analysis to identify the possible key stakeholders for this project. As previously mentioned in the literature review, a diverse group of stakeholders is essential to ensure that the currrent situation is reviewed from multiple perspectives. 

Afterwards, the next step is to utilise the chosen stakeholders to gather as much information as possible from various perspective to ensure that the project is aligned with the needs of both patients and healthcare professionals in Moldova and solves the existing problems.

\section{Stakeholder Analysis}

The student has identified the following possible stakeholders for the project:

\begin{itemize}
    \item Doctors and other medical staff working in hospitals
    \item Department head in a hospital
    \item IT staff members in hospitals
    \item Staff members at CNAM (National Health Insurance Company)
    \item Staff members at the Ministry of Health 
    \item Patients 
\end{itemize}

These stakeholders have been identified so that they can provide a wider picture on the needs and requirements of the project, and to ensure that the project is aligned with the expectations of workers within the healthcare industry in Moldova from multiple perspectives. 

The stakeholders have also been placed in the stakeholder influence-interest grid (which will be discussed in section \ref{sec:stakeholders}) to help the student understand the level of influence and interest that each stakeholder has in the project:

\begin{figure}[ht]
    \centering
    \includegraphics[scale=0.5]{Stakeholder_analysis1.png}
    \caption{Chosen stakeholders in the stakeholder influence-interest grid}
    \label{fig:stakeholder_analysis1}
\end{figure}

\subsection{Current situation analysis}

Following the analysis, the student has conducted several exploratory interviews with the chosen stakeholders to gather insights into current issues with the IT systems used in Moldova's healthcare sector. The student was able to reach out to every stakeholder, except for staff members at CNAM and the Ministry of Health. 

After the conclusion of the interviews, three main issues and potential solutions have been identified:

\begin{enumerate}
    \item Current EHR system is outdated and only accessible via Internet Explorer or legacy version of Microsoft Edge. A potential solution is to develop a new, modernized version of the existing system (retaining the core functionality) that is accessible via modern browsers, is more user friendly and has future upgrade capabilities.
    \item Lack of interoperability between medical institutions due to a lack of a nationally-wide integrated system. A potential solution is to create a new system that acts as a patient history archive, where patients can upload their own medical records (such as lab tests, previous medical history, etc) and share them with any medical practitioner, regardless of the institution they work at. 
    \item Lack of digitalizion for some systems that still rely on paper-based records or very rudimentary data structures, such as the national transplant registry. A potential solution is digitalized of said system, as is in the case of the transplant registry, that can be accessed by any medical practitioner in Moldova.
\end{enumerate}

Analysing the current issues and potential solutions, the student has determined that the solutions for issues \#1 and \#3 are too complex, as they require a complete overhaul and integration with existing systems. As such, the student has decided to focus on issue \#2, as it is the most feasible and can be implemented within the timeframe of the project. 

Consequently, the stakeholder list has been updated to reflect the changed focus of the project:
\begin{itemize}
    \item Doctors working in hospitals and clinics
    \item Staff members at the Ministry of Health
    \item Staff members at CNAM
    \item Patients that are using both public and private healthcare institutions
    \item Other medical staff members (nurses, pharmacists, etc)
    \item Staff members at CNPDCP (National Center for Personal Data Protection)
\end{itemize}

At the same time, the stakeholder influence-interest grid has been updated to reflect the changes in the project focus:

\begin{figure}[ht]
    \centering
    \includegraphics[scale=0.5]{Stakeholder_analysis2.png}
    \caption{Updated stakeholders in the stakeholder influence-interest grid}
    \label{fig:stakeholder_analysis2}
\end{figure}

\clearpage

\section{Requirements}

After the new stakeholders were identified, additional interviews were conducted to focus on the requirements for the chosen solution and enough information was gathered from the other stakeholders to identify the main requirements for the project. The student was unable to reach out to the staff members at CNAM, the Ministry of Health and CNDCP - instead legislation and regulations on their websites were reviewed \parencite{CNAM,CNPDCP,ministry}. 

All of the gathered requirements can be found in the appendix (section \ref{sec:requirements}), but the most important requirements have been summarized in the table below:

\begin{table}[h!]
    \centering
    \begin{tabular}{|c|c|p{0.65\textwidth}|}
    \hline
    \textbf{ID} & \textbf{Category}                  & \textbf{Requirement}                                                                \\ \hline
    1   & Non-functional     & The system must be accessible on all modern desktop and mobile-based browsers.       \\ \hline
    2   & Non-functional     & The system must store the data in a secure manner, ensuring that only the patient and the doctor can access the data. \\ \hline
    3   & Document upload                 & The system must allow patients to upload their own medical records in a variety of formats (PDF, DOC, etc). \\ \hline
    4  & Personal cabinet        & The system must display the patient's history in a chronological order in the form of a timeline. \\ \hline
    5   & Personal cabinet        & The patient personal cabinet must provide an overview of the patient's history through 3 main sections: personal information, lab tests, and doctor consultations. \\ \hline
    6  & Personal Cabinet & The system must allow patients to add their own personal information, such as name, date of birth, or address. \\ \hline
    7  & Personal Cabinet & The system must allow patients to add their own allergies and vaccinations. \\ \hline
    8   & Shareable link          & The system must allow the patient to generate a shareable link to provide access to their medical records. \\ \hline
    9   & Doctor view & When shared with the doctor, the system must allow the doctor to only view the patient's history, not edit it. \\ \hline
    10   & Patient medication              & The system must allow patients to enter their current medication including details such as the name of the drug, dosage, frequency and start/end date. \\ \hline
    \end{tabular}
    \caption{Summary of the Most Important Requirements}
\end{table}
    
    
    
    