% Title, Author, Abstract, Acknowledgement, Table of Content, etc....

% Front Page begins

\thispagestyle{empty}

\fancypage{}{\fbox}

%\thisfancyput{%
\begin{center}

%Substitute with the right information
\Large{
\hfill \begin{tabular}{l}
ITMB \\
%Replace this text with your degree name, e.g., Compute Science, Computing and Management, Information Technology and Management of Business, Compute Science and Mathematics, Compute Science and Artificial Intelligence
COC252 \\
%e.g COC251, COC252, COC253, COC255, COC257, COC800, COD290 
F011321
%Replace this text with your ID number
\end{tabular}
}


%\bigskip
%\bigskip
\vspace*{\fill}

%replace by your Project title
\Large{\textbf{EHR System Modernisation \\
in the Republic of Moldova}}

\vspace*{\fill}

by

\vspace*{\fill}

%Replace by your name
Calin Corcimaru


%Replace by your supervisor's name
\vspace*{\fill}
Supervisor: Dr.\ Georgina Cosma
\vspace*{\fill}

\underline{Department of Computer Science} \\
\underline{Loughborough University}

\vspace*{\fill}
%Delete/Change as appropriate
May 2025

\end{center}
%}

% Front Page ends

%Reset so that next pages do not have a box around them
\fancypage{}{}

%set roman numbering for the initial pages
\pagenumbering{roman}

% Abstract
\chapter*{Abstract}
\addcontentsline{toc}{chapter}{Abstract}
This report outlines the result of a project that addresses a key issue in Moldova's healthcare sector: fragmentation of patient data across multiple healthcare institutions. The lack of a nationally-wide integrated system has resulted in overreliance of paper-based and manual processes, leading to reduced patient care quality and incomplete medical histories. Following initial stakeholder consultations and a review of the relevant literature, the project proposed a patient-focused prototype  that allowed users to collect and manage their patient data. The solution enables document uploading and record creation, automated lab result extraction with multimodal large language models and a controlled sharing mechanism with healthcare practitioners. The prototype was implemented using a Vue frontend and FastAPI backend as a web application that demonstrates the feasibility of a patient-focused solution in Moldova's healthcare sector. Stakeholder feedback was collected throughout the whole development process through a series of live demonstrations, allowing for iterative improvements to the application. Final stakeholder feedback confirms the application's potential to enhance care quality and introduce a patient-centric approach to healthcare in Moldova. Overall, this project aims to contribute to Moldova's digital healthcare transformation and hopes to serve as a foundation for future development and implementation of a real-world service.

\textbf{Keywords:} Personal Health Record, Republic of Moldova, Data Fragmentation, Multimodal Large Language Models, Patient-Centric Healthcare

% Acknowledgements
\chapter*{Acknowledgements}
\addcontentsline{toc}{chapter}{Acknowledgements}
I would like to express my gratitude to my project supervisor, Dr.\ Georgina Cosma, for her invaluable guidance and support throughout this project. Her feedback and encouragement have been crucial in motivating and encouraging me to continue exploring and innovating in my system design and development.

I am also deeply grateful to all the stakeholders who have generously contributed their time and ideas to this project. Their insights into the workings of Moldova's healthcare sector and the digital landscape of healthcare technology have been invaluable in shaping the direction of this project. Their feedback and suggestions have been instrumental in refining the system design and ensuring that it meets the needs of its users and improves the quality of care provided to patients. This ultimately ensured that this project efficiently addresses the real-world challenges faced by the healthcare sector in Moldova.

Next, I would like to extend my appreciation to the Loughborough University staff and faculty who provided all the necessary knowledge and support throughout my studies. This project represents a culmination of everything I have learned during my degree, and I am grateful for the opportunity to apply my skills and knowledge in a practical setting.

Finally, I would like to thank my family and friends, who have always been encouraging me during this project. Their belief in my abilities and support have been a source of motivation and inspiration.

I'd like to dedicate this project to my late grandfather, Ion Corcimaru, who was a well-known doctor in Moldova. His devotion to this field and commitment to saving patient lives have always been a source of inspiration for our family. Even though I'm not pursuing a career in medicine, I hope to honour his legacy by contributing to the improvement of healthcare in Moldova through this project. I hope that my work will help to create a better healthcare system for future generations, just as he did for his patients.


% Set the depth for your table of content
% Currently set at 2 (Chapter, Section, Subsection)
\setcounter{tocdepth}{2}

% Include a table of content
\tableofcontents
\addcontentsline{toc}{chapter}{Table of Contents}

% Include a table of figures
\listoffigures
\addcontentsline{toc}{chapter}{List of Figures}

% Include a table of tables
\listoftables
\addcontentsline{toc}{chapter}{List of Tables}

% List of abbreviations
\chapter*{List of Abbreviations}
\addcontentsline{toc}{chapter}{List of Abbreviations}

\begin{tabular}{p{0.2\textwidth} p{0.7\textwidth}}
API & Application Programming Interface \\
CNAM & Centrul Național de Asigurări în Medicină (National Health Insurance Company) \\
CNPDCP & Centrul Național pentru Protecția Datelor cu Caracter Personal (National Center for Personal Data Protection) \\
EHR & Electronic Health Record \\
EMR & Electronic Medical Record \\
HTTP & Hypertext Transfer Protocol \\
HTTPS & Hypertext Transfer Protocol Secure \\
JSON & JavaScript Object Notation \\
JWT & JSON Web Token \\
LLM & Large Language Model \\
MLLM & Multimodal Large Language Model \\
MFA & Multi-Factor Authentication \\
NLP & Natural Language Processing \\
ORM & Object-Relational Mapping \\
PHR & Personal Health Record \\
SDLC & Software Development Life Cycle \\
SFC & Single File Component \\
SIA AMS & Sistemul informațional automatizat „Asistența Medicală Spitalicească” \\
SPA & Single Page Application \\
SSL & Secure Sockets Layer \\
UML & Unified Modeling Language \\
USMF & Universitatea de Stat de Medicină și Farmacie (State University of Medicine and Pharmacy) \\
UUID & Universally Unique Identifier \\
\end{tabular}

% Glossary
\chapter*{Glossary}
\addcontentsline{toc}{chapter}{Glossary}

\begin{description}
    \item[Application Programming Interface (API)] A set of protocols and tools that allow different software applications to communicate with each other.
    \item[Authentication] The process of verifying the identity of a user.
    \item[Authorisation] The process of determining whether a user has permission to access a resource or perform an action.  
    \item[Backend] The server-side of an application that handles business logic, database operations and API requests.
    \item[Dependency injection] A software engineering design pattern where a function receives the object it depends on from an external source rather than creating it itself.
    \item[Entity-Relationship Diagram (ERD)] A visual representation of the database structure, showing the relationships between different entities.
    \item[Frontend] The client-side of an application that users interact with by using a user interface. 
    \item[Hallucination] A phenomenon in which a large language model generates information that appears to be factual but is incorrect or completety made up.
    \item[Javascript Object Notation (JSON)] An interchange format for structured data representation, often used in APIs to exchange data between a client and a server.
    \item[Object Relational Mapping (ORM)] A programming technique that allows developers to interact with a database by mapping database tables to classes and rows to objects.
    \item[Republican hospital] A tertiary healthcare institution in Moldova that serves as the highest level of specialised medical care in the country.
    \item[Single Page Application (SPA)] A web application that loads a single page and dynamically updates the content as the user interacts with the app.
\end{description}