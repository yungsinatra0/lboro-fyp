\chapter{Project Planning and Design}

\section{Project Management Methodology and Tooling}

Based on the research above, the student has decided that he will be using a hybrid approach, with Waterfall as the main methodology for planning managing the project. The development part of the project will be done using ScrumBan, so that the student will be able to utilise elements from both frameworks. There are several reasons for this choice:

\begin{enumerate}
    \item The nature of the project - the student is working on a project that has a limited timeframe (about 6-7 months) and is of a smaller scale. 
    \item Documentation requirements - the student is required to document the progress during the project in this report, including the requirements gathered, design considerations and implementation decisions and outcomes. 
    \item Regulatory requirements - the student is required to adhere to the regulations and standards of the healthcare industry, which may require extensive documentation and planning.
    \item Customer involvement - the student will be working closely with the project stakeholder, who will be providing feedback and guidance throughout the project.
    \item Familiarity with both Agile and Waterfall - the student has experience with both Agile (specifically Scrum and Kanban) and Waterfall methodologies, and has worked on projects that have used both approaches.
\end{enumerate}

The student will use Jira Software as their project management tool, which is one of the most popular project management tool for software development projects that supports working with Agile frameworks such as Scrum and Kanban \parencite{atlassian}.

\section{Requirements and Stakeholder Analysis}

To gain a more complete understanding of the current situation in Moldova, the existing problems and possible needs of the people involved, it is important to start with an analysis to identify the possible key stakeholders for this project. As previously mentioned in the literature review, a diverse group of stakeholders is essential to ensure that the currrent situation is reviewed from multiple perspectives. 

Afterwards, the next step is to utilise the chosen stakeholders to gather as much information as possible from various perspective to ensure that the project is aligned with the needs of both patients and healthcare professionals in Moldova and solves the existing problems.

\subsection{Stakeholder Analysis}

The student has identified the following stakeholders for the project:

\begin{enumerate}
    \item A doctor working at the Republican Hospital - the main stakeholder, who will be providing guidance and feedback throughout the project on the perspective of the end users, the doctors and nurses who will be using the EHR system.
    \item The Vice-Director of CNAM (National Health Insurance Company) - the person who will be providing feedback from the perspective of the health insurance company.
    \item Senior IT staff member at CNAM - a person who also has previously worked within the Ministry of Health, and who will be providing feedback on the technical aspects of the integration between the EHR system and the health insurance system.
    \item Department head at the Republican Hospital - a person who will be providing feedback on the administrative aspects of the EHR system.
    \item Patients - both of public and private healthcare institutions, who will be providing feedback on the usability and accessibility of an EHR system prototype.
\end{enumerate}

These stakeholders have been identified so that they can provide a bigger picture on the needs and requirements of the project, and to ensure that the project is aligned with the needs of the healthcare industry in Moldova from both the provider and patient perspective. 

The stakeholders have also been placed the stakeholder influence-interest grid, to help the student understand the level of influence and interest that each stakeholder has in the project:

\begin{enumerate}
    \item The doctor working at the Republican Hospital - low influence, high interest
    \item The Vice-Director of CNAM - high influence, high interest
    \item Senior IT staff member at CNAM - high influence, low interest
    \item Department head at the Republican Hospital - high influence, low interest
    \item Patient - low influence, high interest
\end{enumerate}

\subsection{Current situation analysis}

Following the stakeholder analysis, the student has conducted several exploratory interviews with the chosen people to provide insights into the current issues with the IT systems used in the healthcare sector. After the conclusion of the interviews, several main themes for problems and potential solutions have been identified:

\begin{enumerate}
    \item Current system is outdated, not user friendly and only accessible via Internet Explorer or legacy version of Microsoft Edge. A potential solution is to develop a new, modernized version of the existing system (thus retaining the core functionality) that is accessible via modern browsers and devices and can be augmented with additional features if necessary.
    \item Lack of interoperability between medical institutions due to a lack of a nationally-wide integrated system. A potential solution is to create a new systems where patients can upload their own medical records (such as lab tests, previous medical history, etc) and share them with any medical practitioner, regardless of the institution they work at. 
    \item Some systems are not digitalized at all, and still rely on paper-based records or very rudimentary data structures, such as the transplant registry. A potential solution is a the creation of a digitalized system, as is in the case of the transplant registry, that can be accessed by any medical practitioner in Moldova.
\end{enumerate}

After analysing the current issues and potential solutions, the student has determined that the solutions for issues \#1 and \#3 are too complex, as they require an overhaul of the system and integration with other existing systems. As such, the student has decided to focus on issue \#2, as it is the most feasible and can be implemented within the timeframe of the project. 

After the decision has been made, more interviews were conducted with stakeholders to focus on the requirements for the chosen solution. The requirements have been gathered and documented in the next section.

\subsection{Requirements}

\subsection{Product Backlog}

\section{System Design}

\subsection{UML Diagrams}

\subsection{Database Design}

\section{Project Tech Stack}

\subsection{Frontend}

\subsection{Backend}

\subsection{Database}