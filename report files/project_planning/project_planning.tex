\chapter{Project Planning and Design}

\section{Project Management Methodology and Tooling}

Based on the research above, the student has decided that he will be using a hybrid approach, with Waterfall as the main methodology for planning managing the project. The development part of the project will be done using ScrumBan, so that the student will be able to utilise elements from both frameworks. There are several reasons for this choice:

\begin{enumerate}
    \item The nature of the project - the student is working on a project that has a limited timeframe (about 6-7 months) and is of a smaller scale. 
    \item Documentation requirements - the student is required to document the progress during the project in this report, including the requirements gathered, design considerations and implementation decisions and outcomes. 
    \item Regulatory requirements - the student is required to adhere to the regulations and standards of the healthcare industry, which may require extensive documentation and planning.
    \item Customer involvement - the student will be working closely with the project stakeholder, who will be providing feedback and guidance throughout the project.
    \item Familiarity with both Agile and Waterfall - the student has experience with both Agile (specifically Scrum and Kanban) and Waterfall methodologies, and has worked on projects that have used both approaches.
\end{enumerate}

The student will use Jira Software as their project management tool, which is one of the most popular project management tool for software development projects that supports working with Agile frameworks such as Scrum and Kanban \parencite{atlassian}.

\section{Requirements and Stakeholder Analysis}

\subsection{Stakeholder Analysis}

To gain a more complete understanding of the current situation and the needs of the stakeholders, the student has conducted a stakeholder analysis. The student has identified the following stakeholders:

\begin{enumerate}
    \item A doctor working at the Republican Hospital - the main stakeholder, who will be providing guidance and feedback throughout the project on the perspective of the end users, the doctors and nurses who will be using the EHR system.
    \item The Vice-Director of CNAM (National Health Insurance Company) - the person who will be providing feedback from the perspective of the health insurance company.
    \item Senior IT staff member at CNAM - a person who also has previously worked within the Ministry of Health, and who will be providing feedback on the technical aspects of the integration between the EHR system and the health insurance system.
    \item Department head at the Republican Hospital - a person who will be providing feedback on the administrative aspects of the EHR system.
    \item Patients - while they are not directly involved in the project, the EHR system will have a direct impact on them, as it will help improve the quality of care they receive.
\end{enumerate}

These stakeholders have been identified so that they can provide a bigger picture on the needs and requirements of the project, and to ensure that the project is aligned with the needs of the healthcare industry in Moldova. 

The stakeholders have also been placed the stakeholder influence-interest grid, to help the student understand the level of influence and interest that each stakeholder has in the project:

\begin{enumerate}
    \item The doctor working at the Republican Hospital - low influence, high interest
    \item The Vice-Director of CNAM - high influence, high interest
    \item Senior IT staff member at CNAM - high influence, low interest
    \item Department head at the Republican Hospital - high influence, high interest
    \item Patient - low influence, low interest
\end{enumerate}

\subsection{Requirements}

Following the stakeholder analysis, the student has conducted several interviews with the key stakeholders to gather the requirements for the project. The student has identified the following requirements, which are prioritized using the MoSCoW method. The requirements have been categorized in various sections based on the area of the system they are related to. 

At the moment, the requirements are kept at a high level, and will be further refined when they will be moved to the product backlog and then individually during their respective Sprints.

\subsubsection{Accessibility}
\begin{itemize}
    \item The system must be accessible using any desktop-based web browser
    \item The system must be accessible from any location, as long as the user has an internet connection
    \item The system should be accessible on mobile devices using a web browser
    \item The system could allow doctors to access test results and vitals using a mobile web browser
    \item The system could allow doctors to create basic daily record entries using a mobile web browser
\end{itemize}

\subsubsection{Security}
\begin{itemize}
    \item The system must have an audit log that records all changes made to the patient records by the system users
    \item The system must have role-based access control, with different roles for doctors, nurses, and administrative staff
    \item The system must restrict edit access to patient records for other doctors and nurses, except for the doctor who is assigned to the patient
    \item The system must restrict editing of patient history after it was taken by the doctor
\end{itemize}

\subsubsection{Vitals and test results}
\begin{itemize}
    \item The system must allow doctors to view the history of the hospitalized patient's test results in a graphical and numerical format
    \item The system must allow doctors to view the history of the hospitalized patient's vitals in a graphical and numerical format
    \item The system must load the patient's vitals and test results within 3 seconds of the doctor opening the patient's record
    \item The system must allow doctors to order lab tests for the patient
    \item The system should allow doctors to export the patient's test results and vitals in Excel format
    \item The system could allow doctors to generate a barcode to be printed and attached to the patient's sample
\end{itemize}

\subsubsection{Prescriptions}
\begin{itemize}
    \item The system must allow doctors to order medications
    \item The system must allow allow doctors to assign medications to the hospitalized patient
    \item The system must allow doctors and nurses to track the assigned hospitalized patient's medication intake
    \item The system should allow doctors to specify the dosage, frequency, duration and method of administration of the medication
\end{itemize}

\subsubsection{Patient records}
\begin{itemize}
    \item The system must allow doctors to view the patient's previous hospitalization records at the same hospital
    \item The system must allow doctors to create hospitalization-related records for the patient: admission, discharge and transfer
    \item The system must allow doctors to create daily record entries for the hospitalized patient
    \item The system must provide a free text field for doctors to create record entries
    \item The system must store and display the patient's personal information
    \item The system must allow doctors to change the patient's diagnosis after the patient has been discharged
    \item The system could allow doctors to manually upload patient records from paper-based records
    \item The system could allow doctors to build an empty document template by choosing the required fields or sections from a pre-defined list
\end{itemize}

\subsubsection{Insurance}
\begin{itemize}
    \item The system mult allow doctors to view the patient's insurance information
    \item The system must allow doctors to assign an ICD-10 code to the patient's diagnosis
    \item The system must allow doctors to add free text notes to the patient's diagnosis
\end{itemize}

\subsubsection{Administration}

\begin{itemize}
    \item The system must communicate with ASP system to extract patient personal data using IDNP at the time of patient registration
    \item The system must allow administrative staff to register new patients
    \item The system must allow administrative staff to generate analytics reports based on aggreggate patient data
\end{itemize}

\subsubsection{Additional Features}
\begin{itemize}
    \item The system could generate a summary of the previous patient's hospitalization records using artificial intelligence
    \item The system could utilise a speech recognition system to allow doctors to dictate daily record entries
    \item The system could utilise artificial intelligence to provide general recommendations to patients based on their diagnosis
    \item The system could utilise OCR (Optical Character Recognition) to extract data from paper-based records
\end{itemize}

Based on the requirements above, the student has established the following user categories and their respective roles:
\begin{enumerate}
    \item Doctor - the main user of the system, who will be responsible for creating and managing the patient records
    \item Nurse - the user who will be responsible for tracking the patient's medication intake
    \item Administrative staff - the user who will be responsible for registering new patients and generating analytics reports
    \item Patient - will not have direct access to the system, but will contain all the information related to the patient
    \item Pharmacist - the user who will be responsible for approving and dispensing the medications orders
    \item Laboratory technician - the user who will be responsible for conducting the lab tests and entering the results into the system
    \item Insurance company - the user who will be responsible for approving the patient's insurance claims
    \item System administrator - the user who will be responsible for managing the system, including user roles and permissions
\end{enumerate}

\subsection{Product Backlog}

\section{System Design}

\subsection{UML Diagrams}

\subsection{Database Design}

\section{Project Tech Stack}

\subsection{Frontend}

\subsection{Backend}

\subsection{Database}