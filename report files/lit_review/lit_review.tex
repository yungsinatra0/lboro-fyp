\chapter{Literature Review}

This chapter will provide a review of the existing literature, which will be used guide the student in their planning and development efforts of the project.

\noindent As such, it will be covering the following areas:
\begin{itemize}
    \item Software development methodologies
    \item Requirement and stakeholder management
    \item Tech stack (backend and frontend)
    \item Large Language Models (LLMs)
    \item PHR Systems
\end{itemize}

\section{Software development methodologies}\label{sec:methodologies}

\subsection{Software Development Life Cycle}

The Software Development Life Cycle (SDLC) is a process used to guide the development of software applications or systems \parencite{sdlc1}. The SDLC consists of multiple phases, each with its own set of activities and deliverables.`\ '\textcite{sdlc2} outline the phases of the SDLC as following:
\begin{enumerate}
    \item \textbf{Requirement gathering and analysis phase} --- the requirements are gathered saved in a document. Based on the requirements gathered, a development plan is created.
    \item \textbf{Design phase} --- requirements are written in a more technical manner and system desings are created. 
    \item \textbf{Implementation phase} --- Actual development of the software occurs in this phase. Additionally, some smaller unit tests may be done during this phase.
    \item \textbf{Testing phase} --- may involve multiple types of testing, such as unit testing, integration testing, and system testing.`\ '\textcite{testing} describes the different types of tests:
    \begin{itemize}
    \item Unit testing --- testing individual units or components of the software.
    \item Integration testing --- performed on two or more units combined together, focusing on the interfaces between these components.
    \item System testing --- focuses on the `end-to-end quality of the entire system', testing it as a whole.
    \end{itemize}
    \item \textbf{Maintenance phase} --- involves the deployment and maintenance of the software. Additionally, this phase may include end-user acceptance testing, to ensure that it meets their needs \parencite{testing}. 
\end{enumerate}

\begin{figure}[ht]
    \centering
    \includegraphics[scale=0.7]{SDLC.png}
    \caption{Software Development Life Cycle}\label{fig:sdlc}
\end{figure}

\newpage

\subsection{SDLC Models}

The literature describes several SDLC models that have been used in the development of software applications.`\ '\textcite{sdlc1, sdlc2} highlight the most common ones: Waterfall model, V model, Spiral model, Interative model, and Agile model.

\subsubsection{Waterfall Model}

The Waterfall Model is probably the most well-known SDLC model. It is a linear model, where the development process is divided into distinct, sequential phases (which can be seen in figure~\ref{fig:waterfall}).

The Waterfall Model's strengths lie in its simplicty of use, ease of understanding and a clear, structured approach \parencite{waterfall}. An additional stregth that the authors note is its extensive documentation and planning, emphasizing quality and adherence to regulations. 

On the other hand, one of Waterfall's main weaknesses is its lack of flexibility in regards to change \parencite{waterfallno}. Thus, this model is not suitable for projects where the requirements are not well understood or are likely to change. Additionally, the project deliverable is not avaialable until the end of the project, any changed or feedback cannot be done during its development \parencite{waterfallno}.

\begin{figure}[ht]
    \centering
    \includegraphics[scale=0.6]{Waterfall.png}
    \caption{Waterfall model}\label{fig:waterfall}
\end{figure}

\subsubsection{Agile Model}

Another well-known model is Agile. It has multiple frameworks, with Scrum and Kanban being the most popular ones. Scrum is a framework where the project is divided into sprints, each lasting between 2--4 weeks, that aim on delivering value to the customer through incremental software features \parencite{scrumban, agile}. Kanban focuses on visualizing the project workflow by using a visual board with columns, cards and swimlanes. It uses column limits and a pull system to make the flow of work through the system more efficient \parencite{agile}.

Agile has some drawbacks --- its lack of documentation and formal planning, especially in the early stages of the project, may not be suitable for large scale projects \parencite{agile, sdlc2}. Similarly, lack of knowledge on how to use the frameworks may be a barrier for some teams \parencite{waterfallno, sdlc2}.

Nowadays, the combined use of Scrum and Kanban is becoming quite popular, with many teams employing both frameworks in their projects, allowing them to adopt the appropriate practices and adapt them accordingly based on their needs \parencite{scrumban}.

\begin{figure}[ht]
    \centering
    \includegraphics[scale=0.5]{Scrum.png}
    \caption{Scrum framework}\label{fig:scrum}
\end{figure}

\subsection{A hybrid approach}

A hybrid approach has also been emerging in software development projects. Various surveys report the most common combinations are Scrum, Iterative Development, Kanban, Waterfall and DevOps, with hybrid Waterfall and Scrum being the most popular one \parencite{hybrid1,hybrid2}. In this approach, the development part is done in an Agile way, with the rest of the project using Waterfall as a backbone \parencite{hybrid2}.

\textcite{hybrid1} note that projects using either Agile, traditional or hybrid approach show similar levels of success in terms of budget, time and quality. However, the authors have found that agile and hybrid approaches perform much better on customer satisfaction than the traditional ones.

\section{Requirements gathering}

Requirements gathering is the first step in any software development process. As described by \textcite{reqanalysis2}, a requirement is a `necessary attribute in a system\ldots that identifies a capability, characteristic, or quality factor of a system in order for it to have value and utility to a user'. Multiple studies mention how proper requirement gathering plays a pivotal role in the project success, with many project failures attributed to poor requirements gathering \parencite{reqanalysis1, reqanalysis3, reqanalysis5}.

\subsection{Requirement types}

Requirements can be classified into 2 categories: functional and non-functional. Functional requirements describe the system's behavior, while non-functional requirements describe the system's quality attributes, such as performance, security, reliability, etc.`\ '\parencite[6]{requirements}.

When writing the requirements in a document, it is important to ensure clarity and conciseness to avoid ambiguity. Based on the recommendations of \textcite[112]{requirements} and \textcite{requirements2}, the following principles and practices can be followed:  
\begin{itemize}
    \item Write in a simple and consistent language.
    \item Avoid technical jargon, vague terms, and combining multiple requirements in a single statement.
    \item Ensure that requirements are necessary, appropriate, complete, feasible, and verifiable.
    \item Include attributes for each requirement, such as identification, owner, priority, risk, rationale, difficulty, and type (functional/non-functional).
\end{itemize}

Prioritizing requirements is another crucial task, especially in projects with numerous requirements. Some methods mentioned by \textcite{moscow} include using a low to high priority, assigning a numerical value within a specific range or MoSCoW, which classifies requirements into four categories: 
\begin{itemize}
    \item Must have --- must be implemented in the software before being released
    \item Should have --- important but not necessary for the software to be released
    \item Could have --- desirable but not necessary for the software to be released
    \item Won't have --- requirements that are not included in the current release
\end{itemize}

\subsubsection{Requirements in Agile}

In Agile projects, requirements are written in the form of User Stories, which are simple descriptions of a feature desired by the customer, using a specific format: `As a [user], I want to [action] so that [benefit]' \parencite[191]{requirements}. Components of a User Story include a title, acceptance criteria, priority, story points and description. Epics are requirements that cannot be completed in a single sprint and can be broken down into user stories. Epics and User Stories are part of the Product and Sprint backlogs, which contain the requirements for the whole project and the current sprint, respectively.

\subsection{Stakeholders}\label{sec:stakeholders}

Stakeholders are the individuals who have some interest in the success of the system or project, thus it is important to identify all possible stakeholders in the early stages of the project to avoid missing important requirements or constraints \parencite[34]{requirements}. 

Stakeholder analysis can help understand their position within the project. One way of doing it is by using a stakeholder matrix, such as the Influence/Interest grid (see figure~\ref{fig:stakeholder_matrix}), which classifies stakeholders based on their influence and interest in the project \parencite{stakeholders,stakeholders2}.

\begin{figure}[ht]
    \centering
    \includegraphics[scale=0.5]{Stakeholder.png}
    \caption{Stakeholder Influence/Interest matrix}\label{fig:stakeholder_matrix}
\end{figure}

\subsection{Requirement gathering techniques}

Multiple studies mention the most popular requirement gathering techniques are interviews, workshops, prototyping, modelling, brainstorming, storyboards and observing users \parencite{reqanalysis1,reqanalysis2, reqanalysis3, reqanalysis4}. In one of them, individuals with multiple years of experience in requirement gathering were interviewed, and the authors found that the most used techniques were collaborative meetings, interviews, ethnography and modelling \parencite{reqanalysis1}.

Multiple research papers recognise interviews as the most common technique for requirement gathering \parencite{interviews5,interviews1,interviews2}. Some studies have looked at best practices and common mistakes when conducting requirement gathering interviews. The recommended practices, based on \textcite{interviews4, interviews3}, and the common mistakes, from \textcite{interviews1, interviews2}, are summarized in Table~\ref{tab:comparison_mistakes_practices} below.

\begin{table}[h!]
    \centering
    \small
    \renewcommand{\arraystretch}{1.2}
    \begin{tabular}{|>{\arraybackslash}m{0.45\textwidth}|>{\arraybackslash}m{0.45\textwidth}|}
    \hline
    \textbf{Common Mistakes} & \textbf{Recommended Practices} \\ \hline
    Wrong opening: failing to understand the context before discussing the problem. & State goals at the beginning and allow customer input at the end.`\ '\\ \hline
    Not leveraging ambiguity to reveal knowledge gaps. & Avoid ambiguity by asking clarifying questions.`\ '\\ \hline
    Lack of planning: unstructured sequence of questions. & Plan interviews with a structured sequence of questions.`\ '\\ \hline
    Failing to build rapport with the customer. & Building rapport through small talk or personal questions in the beginning.`\ '\\ \hline
    Implicit goals: failing to ask or clarify stakeholder goals. & Verify alignment and current interpretation with the customer's vision.`\ '\\ \hline
    Question omission: not asking about business processes or doing follow-up questions. & Be flexible by probing into relevant topics.`\ '\\ \hline
    Weak communication: too much technical jargon usage or not listening to the customer. & Use projective techniques like scenarios to encourage deep thinking.`\ '\\ \hline
    Poor question formulation: vague, technical, irrelevant, or too long. & Break down questions or responses into smaller parts and use story telling.`\ '\\ \hline
    Wrong closing sequence: skipping interview summaries or feedback. & Leaving time at the end for the stakeholder to offer any feedback or thoughts.`\ '\\ \hline
    \end{tabular}
    \caption{Comparison of Common Mistakes and Recommended Practices}\label{tab:comparison_mistakes_practices}
\end{table}

\section{Tech stack}

\subsection{Database}\label{sec:database}

There are several types of databases, such as: relational (SQL), NoSQL databases, graph databases or object-oriented databases \parencite{databases2}. The choice of database depends on the project or organisation requirements, such as the amount of data, the complexity of the data, the need for scalability, etc. 

\subsubsection{Relational databases}

Relational databases store structured data in tables, linked through keys to create relationships between entries \parencite{databases}. They use SQL (Structured Query Language) to create queries and schemas to help manage data efficiently. \textcite{databases} highlight that relational databases are used, thanks to their high data integrity, for industries like finance and healthcare. Relational databases are widely used, making it easier to find support and resources. However, the rigid schema limits adaptability to rapid data changes or usage of unstructured data. Examples include MySQL, PostgreSQL, Oracle, and Microsoft SQL Server.

\subsubsection{NoSQL databases}

NoSQL databases manage unstructured or semi-structured data without rigid schemas or relationships \parencite{databases}. As the authors describe, NoSQL databases, such as key-value, document, column-family, and graph databases, excel in flexibility and scalability. NoSQL databases often prioritize performance over strict consistency, making them suitable for large datasets of unstructured data but less ideal for complex transactions. NoSQL systems lack SQL’s mature standardization and support. Examples include MongoDB, Cassandra, Couchbase, and Redis.

\subsection{Backend framework}\label{sec:backend}

Choosing the right backend framework is crucial --- it is responsible for handling the business logic of the application, such as processing requests, interacting with the database, and returning responses to the client. A good framework also comes with the added benefit of included features such as security and authentication, database support, a big community and documentation. 

Based on recent a recent survey by \textcite{statista-webframeworks}, the most popular backend frameworks are Express, Flask, Spring Boot, Django and Laravel. This data is also supported by the Stack Overflow Developer Survey 2024, which lists the most popular programming languages as JavaScript (Express), Python (Flask, Django), Java (Spring Boot) and PHP (Laravel) \parencite{stackoverflow}. 

Due to their high popularity among developers, the above frameworks will be compared. Using the information from \textcite{spring,express,django,fastapi}, table~\ref{tab:backend} will compare the frameworks based on the following criteria: programming language used, learning curve, community support, security features, database features, and project size suitability.

\begin{table}[h]
    \centering
    \resizebox{\textwidth}{!}{%
    \begin{tabular}{|l|l|l|l|l|}
    \hline
        \textbf{Framework} & \textbf{Django} & \textbf{FastAPI} & \textbf{Spring Boot} & \textbf{Express} \\
    \hline
        \textbf{Language} & Python & Python & Java & JavaScript \\
    \hline
        \textbf{Learning curve} & Medium & Low & High & Low \\
    \hline
        \textbf{Community Support} & High & High & High & High \\
    \hline
        \textbf{Security features} & High & High & High & Medium \\
    \hline
        \textbf{Database features} & Medium & Low & High & Medium \\
    \hline
        \textbf{Project size suitability} & Small to medium & Small to medium & Medium to large & Small to medium \\
    \hline
    \end{tabular}%
    }
    \caption{Comparison of backend frameworks}\label{tab:backend}
\end{table}

\subsection{Frontend framework}\label{sec:frontend}

Similar to the backend frameworks, choosing a suitable frontend framework is equally important --- it is responsible for the user interface of the application, such as displaying data, handling user interactions, and making requests to the backend. 

Based on the same survey by \textcite{statista-webframeworks}, the most popular frontend frameworks are React, Angular, Vue.js, and Svelte. This data is also supported by the Stack Overflow Developer Survey 2024, where those frameworks rank among the highest for desirability and admirability among developers \parencite{stackoverflow}.

Table \ref{tab:frontend} will use the information from \textcite{react,angular,vue,svelte} to compare the frameworks based on the following criteria: learning curve, community and documentation, ecosystem and tooling support, performance, state management, and project size suitability.

\begin{table}[h]
    \centering
    \resizebox{\textwidth}{!}{%
    \begin{tabular}{|l|l|l|l|l|}
    \hline
        \textbf{Framework} & \textbf{React} & \textbf{Angular} & \textbf{Vue.js} & \textbf{Svelte} \\ 
    \hline 
        \textbf{Learning curve} & Low & High & Medium & Low \\ 
    \hline
        \textbf{Community and documentation} & High & High & High & Medium \\ 
    \hline
        \textbf{Ecosystem and tooling support} & High & High & Medium & Low \\ 
    \hline
        \textbf{Performance} & High & Medium & Medium & High \\ 
    \hline
        \textbf{State management} & High & High & Medium & Low \\ 
    \hline
        \textbf{Project size suitability} & Small to large & Medium to large & Small to medium & Small to medium \\ 
    \hline
    \end{tabular}%
    }
    \caption{Comparison of frontend frameworks}\label{tab:frontend}
\end{table}

\section{Large Language Models (LLMs)}

Large language models (LLMs) are artificial intelligence systems that are used for natural language processing (NLP) tasks such as text generation, translation, summarization and question answering \parencite{llm2,llm_healthcare}. Additionally, LLMs have been found to have emergent capabilities, like reasoning, planning, decision-making and in-contenxt learning \parencite{llm2}. These extraordinary capabilities are achieved through extensive training on large corpus of text data, high parameter count (in the billions) and usage of techniques such as fine-tuning or prompt engineering to improve their performance \parencite{llm2,llm_healthcare}.

LLMs are built on the transformer architecture, which allows them to understand text by learning and remembering the relationships between words \parencite{llm}. These models are first pre-trained on large amounts of unlabeled data using, allowing them to excel in a wide variety of tasks \parencite{foundation, llm2}. These pre-trained models, known as foundation models such as the GPT or Llama families, can then be fine-tuned for specific tasks, improving their performance and accuracy even further \parencite{gpt4,llama3,llm2}.

\subsection{Multimodal LLMs}

One advancement in the field of LLMs has been the addition of multimodal abilities, allowing them to process, understand and generate text and images, audio or videos \parencite{mllm, mllm2}. These new multimodal LLMs (MLLMs) utilise existing reasoning capabilities of LLMs, which are connected to an encoder that can processes images, audio or videos and a generator that helps with generating multimodal outputs \parencite{mllm}. This integration of new modalities allows MLLMs to become versatile tools, expanding their possible use cases and bridging the gap between human and machine interaction \parencite{llm_healthcare}.

\subsubsection{API model providers}

Running and hosting LLMs locally can be a challenge, considering their big model sizes and high computational requirements. As such, many platforms offer APIs that allow users to access LLMs through the cloud. A list of some free API providers, the models offered and their rate limits has been compiled by \textcite{llmapi} and some are listed in the table below.

\begin{table}[h!]
    \centering
    \begin{tabular}{p{2cm} p{5cm} p{6cm}}
        \toprule
        \textbf{Provider} & \textbf{Model name(s)} & \textbf{Free tier limits} \\
        \midrule
        \raggedright
        Groq & Llama 3.2 11B Vision & 7,000 requests/day, 7,000 tokens/minute \\
        & Llama 3.2 90B Vision & 3,500 requests/day, 7,000 tokens/minute \\
        \hline
        \raggedright
        OpenRouter & Llama 3.2 11B Vision Instruct &  \\
        & Llama 3.2 90B Vision Instruct & 20 requests/minute, 200 requests/day \\
        & Gemini 2.0 Flash Experimental &  \\
        \hline
        \raggedright
        Google AI Studio & Gemini 2.0 Flash & 4,000,000 tokens/minute, 10 requests/minute \\
        & Gemini 1.5 Flash & 1,000,000 tokens/minute, 1,500 requests/day, 15 requests/minute \\
        & Gemini 1.5 Pro & 32,000 tokens/minute, 50 requests/day, 2 requests/minute \\
        \hline
        \raggedright
        GitHub Models & OpenAI GPT-4o & Rate limits dependent on Copilot subscription tier \\
        & OpenAI GPT-4o mini & \\
        \hline
        \raggedright
        Cloudflare Workers AI & Llama 3.2 11B Vision Instruct & 10,000 tokens/day \\
        \hline
        \raggedright
        glhf.chat & Any model on Hugging Face that fits on an A100 node (~640GB VRAM)& 480 requests/8 hours \\
        \bottomrule
    \end{tabular}
    \caption{API providers for LLMs}\label{tab:llm_apis}
\end{table}

\FloatBarrier{}
\clearpage

\subsection{LLMs in healthcare}

One application of MLLMs is in healthcare, where the growing volume and complexity of data creates the need for more advanced tools to process and analyze it. LLMs and MLLMs have found use in various healthcare applications, either by using existing models or by developing new, specialized medical models such as Med-PaLm2, BioMistral or Med-Gemini \parencite{biomistral,medgemini,medpalm2}. Some of these applications include:

\begin{itemize}
    \item \textbf{Improving medical diagnosis:} By combining patient records, existing symptoms, and medical history, LLMs can use their reasoning capabilities and memory to assist in diagnosing or preventing health conditions \parencite{llm_healthcare,llm_healthcare3,llm_healthcare4}.
    \item \textbf{Medical Imaging and Multimodal Capabilities:} In diagnostic imaging, multimodal models can assess both text and images (such as X-rays and MRIs) to offer comprehensive analysis. Clinicians can input medical images and contextual information, making MLLMs valuable assistants in the real-time diagnostic processes \parencite{llm_healthcare3}.
    \item \textbf{Virtual Health Assistants:} LLMs can also be deployed as virtual assistants, helping patients with personalised care and general health inquiries \parencite{llm_healthcare,llm_healthcare3}. Patients in areas with limited healthcare access can benefit from these assistants, which also supports healthcare providers by lightening their workloads.
    \item \textbf{Administrative Support:} LLMs can assist in generating Electronic Health Records (EHRs), allowing healthcare providers to focus more on patient interaction \parencite{llm_healthcare4}. Additionally, they can also help translate complex medical terms into more simple language, assist in administrative tasks, and more.
\end{itemize}

\subsection{Prompt Engineering}\label{sec:prompt}

The success of LLMs depends not only on the model itself --- but also on how it's effectively used by the users, using techniques like prompt engineering, which involves the constant designing and refining of prompts to guide the output of LLMs \parencite{promptmed,prompt2}. Prompts represent instructions given to the model to guide its output, such as providing context, examples, or constraints to the model \parencite{prompt,prompt1,prompt2}. 

There are multiple techniques for prompt engineering, ranging from simple to more advanced. The tables and subsections below outlines some of the most common techniques.

\subsubsection{Zero-Shot Prompting}

Zero-shot prompting are techniques where the LLM is given a prompt without any examples, allowing it to generate an output based on the prompt alone \parencite{prompt1}.

\begin{table}[h!]
    \centering
    \begin{tabular}{p{3cm} p{8cm} p{2cm}}
        \toprule
        \textbf{Technique} & \textbf{Description} & \textbf{Source} \\
        \midrule
        \raggedright
        Role prompting & Assigning a specific role to the LLM in the prompt. The authors note that generally it provides mixed results but may be useful in certain settings.  & \textcite{role1} \\
        \hline
        \raggedright
        Style prompting & Specifying the desired style or tone in the prompt. & \textcite{style} \\
        \hline
        \raggedright
        Emotion prompting & Incorporating phrases of psychological relevance to humans in the prompt. & \textcite{emotion} \\
        \hline
        \raggedright
        Re-reading & Adding the phrase `Read the question again' to the prompt in addition to repeating the question. & \textcite{rereading} \\
        \hline
        \raggedright
        Self-Ask & Prompting the LLM to decide if it needs to ask any follow-up questions for a given prompt. &  \textcite{selfask} \\
        \bottomrule
    \end{tabular}
    \caption{Zero-Shot Prompt Techniques}\label{tab:zero_shot}
\end{table}

\FloatBarrier{}

\subsubsection{Few-Shot Prompting}

Few-shot prompting are techniques where the LLM learns how to complete a task based on a few examples given in the input prompt \parencite{prompt1}.

\begin{table}[h!]
    \centering
    \begin{tabular}{p{3cm} p{8cm} p{2cm}}
        \toprule
        \textbf{Technique} & \textbf{Description} & \textbf{Source} \\
        \midrule
        \raggedright
        Self-Generated In-Context Learning & Using the LLM to automatically generate examples when training/example data is not avaialable. & \textcite{self-generating} \\
        \hline
        \raggedright
        Prompt Mining & Scanning the training data to discover common formats that can be used as prompt templates. & \textcite{mining} \\
        \bottomrule
    \end{tabular}
    \caption{Few-Shot Prompt Techniques}\label{tab:few_shot}
\end{table}

\FloatBarrier{}

\subsubsection{Multimodal and Multilingual Prompting}

Multimodal and multilingual prompting are techniques which aim to improve an LLM's performance by leveraging multiple modalities or languages in the prompt \parencite{prompt1}.

\begin{table}[h!]
    \centering
    \begin{tabular}{p{3cm} p{8cm} p{2cm}}
        \toprule
        \textbf{Technique} & \textbf{Description} & \textbf{Source} \\
        \midrule
        \raggedright
        Translate-first prompting & Translating the input prompt into English to leverage LLMs strengths in dealing with English inputs, compared to non-English inputs. & \textcite{translate-first} \\
        \hline
        \raggedright
        English prompting & Writing the prompt in English may usually be more effective than using the task language for multilingual tasks. The authors argue it may because of the predominance of the English language in the pre-training data. & \textcite{english-prompting} \\
        \hline
        \raggedright
        JSON/XML output formatting & Asking the LLM to format the response in a JSON or XML format and providing the expected schema has been found to improve the accuracy of LLM outputs  & \textcite{jsonllm} \\
        \hline
        \raggedright
        Image-as-Text & Generating or writing a textual description of an image that can then be included in a text-based prompt. & \textcite{images-as-text} \\
        \bottomrule
    \end{tabular}
    \caption{Multimodal and Multilingual Techniques}\label{tab:multi_prompt}
\end{table}

\FloatBarrier{}

\subsection{Challenges and concerns of using LLMs}

While LLMs bring many benefits when applied to the healthcare domain, it is important to note that their use does come with several challenges:

\begin{itemize}
    \item \textbf{Data Privacy and Compliance:} Patient data is highly sensitive, thus ensuring compliance with standards is essential, requiring data anonymization and secure handling practices to ensure patient data safety.\parencite{llm_healthcare,llm_healthcare2,llm_healthcare4}.
    \item \textbf{Transparency and Explainability:} LLMs are often described as `black boxes', making it difficult to explain their decision-making processes. In healthcare, transparency is crucial --- lack of it poses risks and raises ethical concerns about relying on such systems in high-stakes scenarios \parencite{llm_healthcare,llm_healthcare2,llm_healthcare4}..
    \item \textbf{Bias and Fairness:} LLMs trained on vast datasets can inherit biases in the data, leading to skewed or unfair outcomes \parencite{llm_healthcare2}.
    \item \textbf{Hallucinations:} LLMs sometimes generate false or fabricated outputs, also known as `hallucinations'. In healthcare, this poses significant risks, as incorrect or misleading information could jeopardize patient safety and trust in the technology \parencite{llm_healthcare4,llm_healthcare}.
    \item \textbf{Accountability:} Responsibility must be clearly communicated and understood by all parties involved in the development and use of the model \parencite{llm_healthcare2}. The author recommends the usage of clear guidelines, policies and code of conducts to ensure that all parties are aware of their obligations.
    \item \textbf{High Costs and Infrastructure Needs:} Training and operating LLMs requires extensive computational resources, which can be a limiting factor for healthcare institutions \parencite{llm_healthcare4}.
\end{itemize}

\section{PHR Systems}

A Personal Health Record (PHR) is an electronic resource
used by patients to manage their own health information \parencite{phrsecurity,phrlist}. PHRs are different from Electronic Health Records (EHRs) and Electronic Medical Records (EMRs) which are inter-organisational or internal systems to organise patient health records \parencite{phrdiff,phrlist}. Three different types of PHRs are described by \textcite{phrsecurity}: stand-alone, which require manual entry to update the records; instituion-specific, which are connected to a specific healthcare institution; and integrated, which can connect to multiple healthcare systems to aggregate data from multiple sources. 

Usage of PHRs can bring many benefits to patients, such as: empowering patients to manage their health, improving patient outcomes, decreasing the cost of healthcare and improving the taking of medication \parencite{phrsecurity}.

PHRs contain highly sensitive health information, so it is important to ensure that the data is secure and private. Based on a survey of health information management and medical informatics experts, \textcite{phrsecurity} identified 7 dimensions that need to be addressed when developing a PHR system:
\begin{enumerate}
    \item Confidentiality
    \item Availability
    \item Integrity
    \item Authentication
    \item Authorization
    \item Non-repudiation
    \item Access rights
\end{enumerate}

The authors recommend mechanisms to ensure adherence to the above-mentioned dimensions, such as encrypting the data in the database, using backups or defining user access to data and access rights.

\subsection{Existing Solutions}

PHR systems have been implemented nation-wide in many developed countries, such as the NHS App in the UK \parencite{phrlist}. Additionally, there are many private solutions that offer similar features to the one proposed in this project. The next sections will provide a brief overview of 3 existing systems: Medvalet, Andaman7 and Fasten Health.

\subsubsection{Medvalet}

A mobile app developed in Romania that allows patients to upload their medical history as PDFs or scanned documents \parencite{medvalet}. See Table \ref{tab:medvalet} for a summary of its features and limitations and figure \ref{fig:medvalet} for a screenshot of the app.

\begin{figure}[h!]
    \centering
    \subfloat[My doctors screen]{\includegraphics[width=0.35\textwidth]{Medvalet_1.png}} \quad
    \subfloat[Uploaded documents screen]{\includegraphics[width=0.35\textwidth]{Medvalet_2.png}}
    \caption{Medvalet screenshots}\label{fig:medvalet}
\end{figure}

\begin{table}[h!]
\centering
    \begin{tabular}{|p{0.47\textwidth}|p{0.47\textwidth}|}
    \hline
    \textbf{Key Features/Benefits} & \textbf{Limitations/Drawbacks} \\ \hline
    \begin{itemize}
        \item Upload medical history as PDFs or scanned documents.
        \item Categorize documents by type (e.g., prescriptions, lab results).
        \item Graphically track vitals like blood pressure and weight over time.
        \item Patients can input personal details such as name, age, and weight.
        \item Doctors can access patient history directly via the app.
    \end{itemize} &
    \begin{itemize}
        \item Requires doctors to create accounts, which may deter use.
        \item Doctors can access patient history without explicit consent, raising privacy concerns.
        \item App acts as document storage, which can be cumbersome to access for lengthy histories.
        \item Lacks data extraction or summarization features from uploaded documents.
        \item Only available as a mobile app, limiting accessibility for desktop-only users.
    \end{itemize} \\ \hline
    \end{tabular}
\caption{Medvalet Features and Limitations}\label{tab:medvalet}
\end{table}

\FloatBarrier{}

\subsubsection{Andaman7}

A mobile app developed by a Belgian-American eHealth company with the goal to improve doctor-patient communication, compliant with GDPR and HIPAA \parencite{andaman}. See Table~\ref{tab:andaman7} for a summary of its features and limitations and figure~\ref{fig:andaman7} for a screenshot of the app.

\begin{figure}[ht]
    \centering
    \subfloat[PHR Sections screen]{\includegraphics[width=0.45\textwidth]{Andaman_1.png}} \quad
    \subfloat[Documents section screen]{\includegraphics[width=0.45\textwidth]{Andaman_2.png}}
    \caption{Andaman7 screenshots}\label{fig:andaman7}
\end{figure}

\begin{table}[htbp]
\centering
    \begin{tabular}{|p{0.47\textwidth}|p{0.47\textwidth}|}
    \hline
    \textbf{Key Features/Benefits} & \textbf{Limitations/Drawbacks} \\ \hline
    \begin{itemize}
        \item Offers sections for personal information, medical history, allergies, vaccinations, medications, etc.
        \item Automatically collects health data from over 300 hospitals and clinics in the US and Europe.
        \item Supports input from diverse sources like hospitals, labs, smart devices or even manual input.
        \item Stores data locally on patients’ devices, ensuring privacy.
        \item Data sharing with QR codes and revokable access.
        \item AI tools for summarization, translation, and simplifying medical jargon.
    \end{itemize} &
    \begin{itemize}
        \item Requires patients and doctors to both create accounts.
        \item Does not extract data or values from uploaded documents like lab results.
        \item Limited to mobile platforms, which may limit usability for desktop-only users.
    \end{itemize} \\ \hline
    \end{tabular}
\caption{Andaman7 Features and Limitations}\label{tab:andaman7}
\end{table}

\FloatBarrier{}

\subsubsection{Fasten Health}

An open-source, self-hosted electronic medical record aggregator with optional paid desktop versions for Windows and Mac \parencite{fasten}. See Table~\ref{tab:fasten_health} for a summary of its features and limitations and figure~\ref{fig:fasten} for a screenshot of the app.

\begin{figure}[ht]
    \centering
    \subfloat[Dashboard screen]{\includegraphics[width=0.75\textwidth]{fasten_1.png}\label{fig:fasten1}} 
    \hspace{0.05\textwidth} 
    \subfloat[Visit history screen]{\includegraphics[scale=0.3]{fasten_2.png}\label{fig:fasten2}} 
    \caption{Fasten Health screenshots}\label{fig:fasten}
\end{figure}

\begin{table}[h!]
\centering
    \begin{tabular}{|p{0.47\textwidth}|p{0.47\textwidth}|}
    \hline
    \textbf{Key Features/Benefits} & \textbf{Limitations/Drawbacks} \\ \hline
    \begin{itemize}
        \item Automatically aggregates records from multiple providers, like hospitals and labs.
        \item Supports self-hosting for complete control over data, stored locally.
        \item Compatible with protocols such as DICOM, FHIR, and OAuth2.
        \item Allows manual entry for allergies, vaccinations, and medications.
        \item Offers multiple dashboards with graphs to visualize health data.
        \item Supports multi-user functionality for families.
    \end{itemize} &
    \begin{itemize}
        \item Paid desktop versions may deter users.
        \item Manual data entry limited to new or existing encounters, complicating usage.
        \item Does not support OCR or automatic data extraction from documents.
        \item Lacks data-sharing capabilities with doctors.
        \item Requires technical expertise for self-hosting.
        \item Restricted to healthcare providers in the United States.
    \end{itemize} \\ \hline
    \end{tabular}
\caption{Fasten Health Features and Limitations}\label{tab:fasten_health}
\end{table}
