\chapter{Conclusion and Future Work}

The final chapter of this report will reflect potential system improvements, explore further development opportunities, analyse key lessons learned and assess the objectives set at the beginning of the project.

\section{Areas of Improvement}

Despite overwhelming stakeholder feedback, several areas were identified to benefit from further improvement before any real-world deployment. These areas include security, compliance, auditing and user experience. The following subsections will briefly discuss these areas and how they can be improved in the future.

\subsection{Security and data encryption}

In its current state, the application incorporates only basic security measures, which would be considered insufficient for a live product. As such, the system would benefit from the following security improvements:

\begin{enumerate}
    \item \textbf{Data at rest encryption:} Currently, all patient information is stored in a plaintext format in the PostgreSQL database. A real-world implementation would require, at a minimum, field-level encryption of sensitive data. If required, additional options such as disk or database-level encryption could be considered.
    \item \textbf{Authentication system enhancements:} The current session-based authentication method would benefit from replacement with a more robust solution, such as a combination of JWT tokens. This could improve security, ensure mobile application compatility and even enable integration with third-party authentication providers like Auth0 or Mpass, which can offer advanced security features like two-factor authentication. 
    \item \textbf{Data in transit encryption:} HTTPS implementation with proper SSL certificate configuration will be an essential step to take before deploying the system to ensure secure communication between the client and server.
\end{enumerate}

\subsection{Logging and monitoring}

The current system implementation lacks any monitoring capabilities. Thus, the integration of a logging system will be a crucial feature for the following reasons:

\begin{enumerate}
    \item \textbf{Error handling:} Logs can be an invaluable resource when it comes to debugging and error handling. This in turn may aid in development and testing efforts, ensuring increased quality of the system.
    \item \textbf{Auditing:} Audit trails are essential for tracking access and changes to sensitive data, such as medical records. This is particularly critical when patients share and potentially edit their data before doctor consultations, ensuring accountability and data integrity.
    \item \textbf{Compliance:} Proper logging and monitoring can help ensure compliance with data protection regulations by providing a clear record of data access and modifications. 
    \item \textbf{Anomaly detection:} Monitoring allows the detection of anomalies in system or user behaviour, which could imply potential security incidents or breaches.
\end{enumerate}

\subsection{Legal and regulatory compliance}

Processing of medical data will require the application to comply with a series of legal and compliance requirements, depending on the user base and system location. At a minimum, a production system will need to ensure the following:

\begin{enumerate}
    \item Compliance with Moldova's data protection and healthcare regulations
    \item Alignment with GDPR (if used by EU citizens)
    \item Creation of a privacy and data protection policy, which will outline how the system collects, uses, stores and processes data
    \item Execution of regular audits and checks to ensure compliance with the above policies and regulations
\end{enumerate}

Additionally, the use of a cloud-based MLLM may introduce additional legal and compliance considerations regarding data processing and storage practices. This may in turn require a transition to locally-hosted models to reduce the risk of data misuse, which may add complexity and cost to the system.

\subsection{User experience improvements}

The current implementation of the user interface prioritised functionality over design and user experience. As such, the system would benefit from a retrofit of the user interface to improve usability and accessibility. This could include:

\begin{enumerate}
    \item \textbf{Native mobile applicationL} In its current state, the system is only available as a web application. A native mobile application would greatly improve accessibility and expand the potential user base, considering Moldova's extensive network coverage \parencite{network}.
    \item \textbf{Accessibility:} Implementation of accessibility features to ensure the system can be used by all users, regardless of their abilities. This could include support for screen readers, keyboard navigation and other assistive technologies.
    \item \textbf{User interface redesign}: While functional, the current user interface lacks a modern design. A redesign could improve the overall user experience, making it more intuitive and user-friendly.
\end{enumerate}

\section{Potential feature expantions}\label{sec:future_work}

Beyond refinement of existing features, the application could significantly benefit from extending its functionality. This section will explore potential additions to the system that could enhance its capabilities and improve the overall user experience. These features were either proposed by stakeholders or considered during the development process but not implemented. 

\subsection{Mpass integration}

Integration with Moldova's governmental electronic authentication and authorization system, Mpass, could provide multiple benefits:

\begin{enumerate}
    \item Simplified user registration and authentication process by reusing existing government-based digital identities
    \item Access to patient demographic information from government databases
    \item Enhanced security through existing identity verification processes, such as electronic signatures
    \item Potential for future integration with other government services
\end{enumerate}

\subsection{Expanding AI features}

The current MLLM implementation is limited to only extracting lab results information from uploaded files. Potential additions that would improve user experience with minimal AI intervention could include:

\begin{enumerate}
    \item Automated processing of uploaded medical files, requiring minimal user input to create a medical history record
    \item Intelligent categorisation of documents based on their content, allowing for smarter organisation of medical records
    \item Intelligent comments and notes generation based on extracted information, providing patients with additional context and insights into their medical history 
\end{enumerate}

\subsection{Integration with lab providers}

The lab results extraction feature currently involves manual uploading of files by the user. A more direct integration with lab providers could streamline this process, allowing for faster and more accurate processing of results. This could be achieved through creating separate, secure email channels or even direct API integration with lab systems. Either option would bring significant benefits, such as:

\begin{enumerate}
    \item Automated results retrieval and delivery to the systems
    \item Elimination of manual entry and file uploads, reducing the risk of human error
    \item Reduced latency between test completion and results availability
\end{enumerate}

\subsection{Enhanced doctor-patient communication}

Late stakeholder feedback identified a potential opportunity to implement a direct communication channel between doctors and patients. Expanding the application with this feature could enable doctors to create new records or add notes to existing records, which would be visible to patients. This would bring the following benefits:

\begin{enumerate}
    \item Reduced risk of miscommunication between doctors and patients during or after consultations
    \item Improved patient engagement and understanding of their medical history and treatment plans
    \item Future potential for telemedicine features, through messages or video consultations
\end{enumerate}

\subsection{Integration with 3rd party providers}

Finally, the system could benefit from integration with 3rd party providers that specialise in health-related services. This could include:

\begin{enumerate}
    \item \textbf{Health tracker apps:} Integration with health tracker applications, such as Google Fit or Apple Health, could allow for automatic syncing of relevant data, providing a more comprehensive view of the patient's medical history.
    \item \textbf{Medical imaging:} Special formats, such as DICOM, are commonly used for medical imaging. As the system currently does not support these formats, integration with specific providers could allow to display X-rays, MRIs and other digital results directly in the application.
\end{enumerate}

\section{Lessons Learned and Reflections}

This section will focus on the valuable insights provided by the project experience that could be applicable to similar initiatives in the future. The lessons learned include key technical and non-technical takeaways from the project, as well as reflections on the overall experience of working on a final year project.

\subsection{Task prioritisation and time management}

A critical lesson learned was the importance of prioritising tasks and managing time effectively, especially on projects with multiple components. The issue occured when the project requirements were not properly prioritised, leading to establishing of a wrong order of implementation. As a result, development began with less critical features, such as vaccines, allergies and medication, rather than the core features like lab extraction or record sharing. This misalignment of priorities resulted in wasted time and resources on less impactful elements, creating time pressure towards the end of the project and limiting the ability to expand the application in areas that required more attention, such as security.

To address this issue in future projects, the following recommendations will be considered:
\begin{enumerate}
    \item Regular priority re-evaluation with team members and stakeholders to ensure a shared understanding of the the future direction of the project
    \item Utilising prioritisation methods such as backlog refinement from Scrum together with priority categorisation approaches like MoSCoW
    \item Aligning initial priorities with stakeholders at the beginning of a project
\end{enumerate}

\subsection{Importance of the team}

Working independently has highlighted the value of teamwork and collaboration in software development. It quickly became apparent that the project would have benefitted from a team of individuals with diverse skills and expertise, such as:

\begin{itemize}
    \item A dedicated business analyst to manage requirements, stakeholder communication and project documentation
    \item A design specialist to optimise the user experience and user interface
    \item A quality assurance engineer to ensure the system's functionality and reliability
    \item A project manager to oversee the project's progress and ensure alignment with timelines and goals
\end{itemize}

This experience has reinforced the importance of collaboration, communication and cross-functional teamwork in software development. While the project was successfully completed, it would have greatly benefitted from the input and expertise of a team of individuals with diverse skills and backgrounds.

\subsection{Agile \& Documentation}

Finally, the project very well illustrated the evolution of documentation in software development, especially in the context of Agile methodologies. Most documents created during the project planning phase experienced significant changes throughout the project due to emerging stakeholder feedback and evolving requirements. This was particularly evident in the UML diagrams, which underwent multiple revisions to reflect the most up-to-date state of the application. It could be said that the documents were `alive', constantly evolving as the project progressed.

As such, this project has demonstrated that effective documentation in software development project should be:

\begin{itemize}
    \item Adaptable, allowing for changes and updates as the project evolves
    \item Regularly updated to reflect the current state of the application
    \item Collaborative, involving input from all team members and stakeholders
    \item Accessible, ensuring that all team members can easily access and understand the documentation
\end{itemize}

\section{Objective assessment}

In the introduction, a set of objectives was established to guide the project. The list can be found in section~\ref{sec:objectives}. This section will assess the extent to which these objectives were met and the overall success of the project.

\begin{enumerate}
    \item Three stakeholders were identified for this project --- 2 healthcare practitioners and 1 patient. Their feedback and previous experience with Moldova's healthcare system was used to inform the design and development of the system.
    \item Initial interviews were conducted with the stakeholders to guide the project's direction. As the project progressed, constant communication was maintained to align with their expectations and requirements.
    \item Three different PHR systems were explored, which provided valuable insights into industry trends and best practices. 
    \item A working web application prototype was developed, which included core features such as lab results extraction, medical record sharing and management.
    \item The final prototype was presented to the client, who expressed interest in creating a start up to further develop the idea. The client also expressed a desire to continue collaborating with the student on this project, which is a positive outcome.
\end{enumerate}

\section{Final Thoughts}

In conclusion, this project provided invaluable practical experience in software development, project and requirements management. It offered a great exposure to the complete software development lifecycle and the challenges that arise at each stage. Additionally, working towards improving the healthcare system in Moldova was a rewarding experience, as it allowed for the application of technical skills to address real-world problems. 

The stakeholder feedback has been praticularly encouraging, with most of them expressing their interest in the continuation and expansion of the project idea. Their offer of continued collaboration presents a future opportunity to transform this project into a real-world solution that could meaningfully improve existing healthcare systems in Moldova.

On a personal level, this experience has been particularly relevant to my future career, providing me with a solid foundation in both technical and non-technical aspects of software development. It has allowed me to developed a transferable skillset, which can be directly applicable to upcoming projects and challenges. As an ITMB student, I feel more prepared than ever to bridge the gap between technology and business, manage complex implementation processes and contribute to the success of any team I may be a part of. 