\chapter{Introduction}\label{chap:introduction}

\section{Background}

The Republic of Moldova is a small country in Eastern Europe that borders Romania and Ukraine, with a current population of 2.4 million people \parencite{mdpop}. Since its independence in 1991, Moldova has faced a number of challenges, including political instability, corruption, and economic difficulties which have left Moldova as one of the poorest countries in Europe \parencite{mdpoverty}. 

Despite these challenges, Moldova has made significant progress in its digital transformation efforts, with the government launching a number of initiatives to modernize its public services and improve the quality of life for its citizens \parencite{mdega}. An example is the Citizen's Government Portal (MCabinet), which allows citizens to access personal information such as `valid identity documents, social contributions and benefits, own properties, information about the family doctor and the health institution where the person is registered, tax payments and other information about the citizen-government relationship' \parencite{mdcabinet}. 

To continue supporting the existing transformation initiatives, Moldova's Cabinet of Ministers has recently approved the `Digital Transformation Strategy of the Republic of Moldova for 2023--2030', which aims to transform the country into a digital society by 2030, with the ultimate goal of having `all public services available in a digitalized format' \parencite{mdstrategy}.

\section{Problem Statement}

The healthcare sector in Moldova has also seen some transformations, with the introduction of a new electronic health record system (EHR) in 15 hospitals across the country in 2017, called `Sistemul informațional automatizat „Asistența Medicală Spitalicească” (SIA AMS)' \parencite{mdehr}. While the system has been successful in helping doctors access patient information more efficiently such as medical history, examinations, test results, and prescriptions, the system hasn't been updated since its inception in 2017 and there are still challenges that need to be addressed in 2024.

The main challenge with the current system lies in the user experience (UX) --- SIA AMS feels old and isn't user-friendly, with a clunky interface that is difficult to navigate, not adhering to modern accesibility standards and only accessible via Internet Explorer or legacy version of Microsoft Edge, with no support for other browsers or devices \parencite{mdehr}. 

Another big challenge with the system is its lack of interoperability within public and private medical institutions due to a lack of a nationally-wide integrated system --- each hospital and clinic have their own, siloed, information system that contains the patient information, with no communication being made between systems in different hospitals \parencite{mdehr}. 

Finally, due to the current economic situation in Moldova, the government has not allocated any funds to upgrade the current or develop new systems, and the hospitals and clinics that use the system do not have the resources to update it themselves.

\section{The Client}

The client, `Nicolae Testemiteanu' State University of Medicine and Pharmacy in Moldova (USMF), is a public university in Chisinau, Moldova, that offers a range of medical programs, including medicine, dentistry and pharmacy \parencite{mduni}. Many of the faculty at USMF are also practicing doctors at hospitals and clinics across Moldova, and have first-hand experience with the current IT systems used in both public and private medical institutions. The USMF faculty members that the student will be interacting with during the project are part of an innovation team that researches potential opportunities to improve the healthcare sector in Moldova through the use of technology. As such, the client has expressed a need for a prototype that can act as a proof of concept for a modern system that could either replace or augment the current system in Moldova.

\section{Project Objectives}

This project aims to initially conduct some research on the current situation of the IT systems used in the healthcare sector in Moldova by interviewing several stakeholders from various healthcare-related institutions. Afterwards, the project will conduct a literature review on the most appropriate technologies and methodologies for developing a modernized EHR system, and an analysis of existing EHR systems to identify their existing functionality. Finally, based on the information gathered, the project will focus on designing and developing a working prototype, based on the requirements gathered and the feasibility of the chosen solution for the Moldovan healthcare sector. The student's hope is that the solution can then be used as a proof of concept to secure funding for a full-scale implementation of the new system in Moldova by the relevant authorities, such as the Ministry of Health.

As such, the objectives of the project are as follows:
\begin{enumerate}   
    \item Identify 2 to 4 stakeholders from various perspectives, such as healthcare institutions in Moldova and patients, that can provide insights into the current IT systems used in the healthcare sector in Moldova.
    \item Conduct interviews with the identified stakeholders to gather information on the current IT systems used in the healthcare sector in Moldova.
    \item Carry out a literature review to research the most appropriate technologies (frontend, backend and database) and project management methodologies for developing a modernized EHR system.
    \item Explore at least 2 existing EHR systems and identify their strengths and weaknesses.
    \item Design and develop a working web or mobile app for an EHR system based on the requirements gathered and the feasibility of the chosen solution for the Moldovan healthcare sector.
    \item Offer the client the prototype to be used as a proof of concept to secure funding for a full-scale implementation of the new system in Moldova.
\end{enumerate}