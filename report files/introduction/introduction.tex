% Set page numbering to arabic for body of Thesis
\pagenumbering{arabic}

\chapter{Introduction}\label{chap:introduction}

\section{Background}

The Republic of Moldova, a small country in Eastern Europe that borders Romania and Ukraine, with a population of 2.4 million people, has faced significant economic and political challenges since gaining independence in 1991 \parencite{mdpop, mdpoverty}. Despite its struggles, Moldova has made substantial progress in its digital transformation initatives aimed at modernising public services and improving citizens' quality of life \parencite{mdega}. A notable example is the Citizen's Government Portal (MCabinet), which provides digital access to all citizens via electronic signature to government-held information and select services \parencite{mdcabinet}. Continuing in this direction, Moldova's Cabinet of Ministers has recently approved the `Digital Transformation Strategy of the Republic of Moldova for 2023--2030', which aims to transform the country into a digital society by 2030, with the ultimate goal of having `all public services available in a digitalized format' \parencite{mdstrategy}.

\section{Problem Statement}

In parallel to government service transformations, the healthcare sector has also seen some digitalisation efforts, evidenced by the introduction of an EHR system in 15 hospitals across the country in 2017, SIA AMS \parencite{mdehr}. While the system has improved practitioners' access to patient information, the system hasn't been updated since its inception in 2017 and several challenges persist in 2024.

The system's user experience represents its main limitation, with an outdated interface that is difficult to navigate and fails to meet modern accessibility standards. Additionally, it is only accessible via Internet Explorer or a legacy version of Microsoft Edge, with no support for other browsers or devices \parencite{mdehr}.

Another fundamental challenge is the siloing of EHR systems across the country. The lack of a nationally-wide integrated system, such as the NHS in the UK, has resulted in fragmented patient data spread across individual healthcare institutions, with no centralised access available \parencite{mdehr}. In turn, this fragmentation has created an overreliance on paper-based or manual processes, which are time-consuming and prone to human error, hindering the quality of care provided to patients.

Moldova's precarious economic situation further complicates any efforts to modernise the current healthcare system. As such, Moldovan citizens are incresingly turning to private digital initiatives and local startups that offer innovative solutions, as evidenced by recent developments showcased at events like the Digital Health Forum 2024 \parencite{mdstartup}.

\section{The Client}

The department of external relations at USMF will serve as this project's client. USMF is a public institution that offers a range of medical programs, including medicine, dentistry and pharmacy \parencite{mduni}. Many of the faculty at USMF are also practicing doctors at hospitals and clinics across Moldova, and have first-hand experience with the current digital systems used in both public and private medical institutions.

Members of the department of external relations have expressed their interest to investigate potential technology-enabled solutions to improve the healthcare sector in Moldova. They have requested for a prototype of a digital system that will either replace or complement existing technological healthcare offerings in Moldova. 

\section{Project Objectives}\label{sec:objectives}

This project aims to address some the challenges faced by the healthcare sector in Moldova, focusing on providing a working, high-fidelity prototype. This will be achieved by a combination of market research and stakeholder interviews, followed by a literature review of existing technologies that will power the prototype, and finally, the design and development of a working prototype. The project's ultimate goal is to create a viable solution that will establish a foundation for potential future broader implementation within Moldova's healthcare sector.

As such, the objectives of the project are as follows:
\begin{enumerate}   
    \item Identify 2 to 4 stakeholders across different healthcare perspectives by the end of November 2024 to provide insights into current practices and systems used in Moldova.
    \item Conduct at least one interview with each identified stakeholder by mid-December 2024 to gather information about current systems used, their challenges and requirements.
    \item Carry out a literature review of at least 30 academic sources by the end of January 2025 to evaluate 3 appropriate frameworks and 2 project management methodologies that can be used to develop a modern EHR system.
    \item Analyse at least 2 existing EHR systems by the end of January 2025 and compare their strengths and weaknesses in a table format.
    \item Design and develop a functional EHR system prototype that implements the top 10 most important gathered requirements by the end of April 2025.
    \item Present the completed prototype to at least 3 stakeholders by end of April 2025 to gather final feedback and validate its potential for further development.
\end{enumerate}