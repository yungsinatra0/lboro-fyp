\chapter{Introduction}
\label{chap:introduction}

\section{Background}

The Republic of Moldova is a small country in Eastern Europe that borders Romania and Ukraine, with a current population of 2.4 million people \parencite{mdpop}. Since its independence in 1991, Moldova has faced a number of challenges, including political instability, corruption, and economic difficulties which have left Moldova as one of the poorest countries in Europe \parencite{mdpoverty}. 

Despite these challenges, Moldova has made significant progress in its digital transformation efforts, with the government launching a number of initiatives to modernize its public services and improve the quality of life for its citizens \parencite{mdega}. An example is the Citizen's Government Portal (MCabinet), which allows citizens to access personal information such as `valid identity documents, social contributions and benefits, own properties, information about the family doctor and the health institution where the person is registered, tax payments and other information about the citizen-government relationship' \parencite{mdcabinet}. 

To continue supporting the existing transformation initiatives, Moldova's Cabinet of Ministers has recently approved the `Digital Transformation Strategy of the Republic of Moldova for 2023-2030', which aims to transform the country into a digital society by 2030, with the ultimate goal of having `all public services available in a digitalized format' \parencite{mdstrategy}.

\section{Problem Statement}

The healthcare sector in Moldova has also seen some transformations, with the introduction of a new electronic health record system (EHR) in 15 hospitals across the country in 2017, called `Sistemul informațional automatizat „Asistența Medicală Spitalicească” (SIA AMS)' \parencite{mdehr}. While the system has been successful in helping doctors access patient information more efficiently such as medical history, examinations, test results, and prescriptions, the system hasn't been updated since its inception in 2017 and there are still challenges that need to be addressed in 2024. 

The first challenge occurs due to a lack of a nationally-wide integrated system -- each hospital and clinic have their own, siloed, information system that contains the patient information, with no communication being made between systems in different hospitals \parencite{mdehr}. 

The second challenge lies with the user experience (UX) of the existing system -- the current system feels old and isn't user-friendly, with a clunky interface that is difficult to navigate, not adhering to modern accesibility standards and being only accessible on legacy version of Edge, with no support for other browsers or devices \parencite{mdehr}. 

Finally, due to the current economic situation in Moldova, the government has not allocated any funds to update the system, and the hospitals and clinics that use the system do not have the resources to update it themselves. 

\section{The Client}

To address these challenges, this project, in collaboration with "Nicolae Testemiteanu" State University of Medicine and Pharmacy in Moldova (USMF), seeks to develop a prototype for a modernized EHR system. The client, USMF, is a public university in Chisinau, Moldova, that offers a range of medical programs, including medicine, dentistry and pharmacy \parencite{mduni}. Many of the faculty at USMF are also practicing doctors at hospitals and clinics across Moldova, and have first-hand experience with the current EHR system in place. As such, the client has expressed a need for a new EHR system prototype that is more user-friendly, accessible on a wider range of devices and browsers, and that can be customized to meet the specific needs of different hospitals, clinics, departments and users.